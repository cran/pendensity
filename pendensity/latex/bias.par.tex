\HeaderA{bias.par}{Calculating the bias of the paramater beta}{bias.par}
\keyword{nonparametric}{bias.par}
\begin{Description}\relax
Calculating the bias of the parameter beta.
\end{Description}
\begin{Usage}
\begin{verbatim}
bias.par(penden.env)
\end{verbatim}
\end{Usage}
\begin{Arguments}
\begin{ldescription}
\item[\code{penden.env}] Containing all information, environment of pendensity()
\end{ldescription}
\end{Arguments}
\begin{Details}\relax
The bias of the parameter beta is calculated as the product of the penalty parameter lambda, the penalzied second order derivative of the log likelihood function w.r.t. beta 'Abl2.pen', the penalty matrix 'Dm' and the parameter set 'beta'.
\deqn{Bias(\beta)= - \lambda  {Abl2.pen(\beta)}^{-1} D_m \beta}{\eqn{Bias(beta)=lambda Abl2.pen(beta)^-1 Dm beta}{}}

The needed values are saved in the environment.
\end{Details}
\begin{Value}
Returning the bias of the parameter beta.
\end{Value}
\begin{Author}\relax
Christian Schellhase <cschellhase@wiwi.uni-bielefeld.de>
\end{Author}
\begin{References}\relax
Penalized Density Estimation, Kauermann G. and Schellhase C. (2009), to appear.
\end{References}

