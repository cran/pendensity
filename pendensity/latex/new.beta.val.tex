\HeaderA{new.beta.val}{Calculating the new parameter beta}{new.beta.val}
\keyword{nonparametric}{new.beta.val}
\begin{Description}\relax
Calculating the direction of the Newton-Raphson step for the known beta and iterate a step size bisection to control the maximizing of the penalized likelihood.
\end{Description}
\begin{Usage}
\begin{verbatim}
new.beta.val(llold, penden.env)
\end{verbatim}
\end{Usage}
\begin{Arguments}
\begin{ldescription}
\item[\code{llold}] log likelihood of the algorithm one step before
\item[\code{penden.env}] Containing all information, environment of pendensity()
\end{ldescription}
\end{Arguments}
\begin{Details}\relax
We terminate the search for the new beta, when the new log likelihood is smaller than the old likelihood and the step size is smaller or equal 1e-3. We calculate the direction of the Newton Raphson step for the known \eqn{beta_t}{} and iterate a step size bisection to control the maximizing of the penalized likelihood \deqn{l_p(\beta_t,\lambda_0)}{\eqn{l(beta,lambda0)}{}}. This means we set \deqn{\beta_{t+1}=\beta_t - 2^{-v} \{s_p(\beta_t,\lambda_0) \cdot (-J_p(\beta_t,\lambda_0))^{-1}\}}{\eqn{beta[t+1]=beta[t]-(2/v)*sp(beta,lambda0)*(-Jp(beta[t],lambda0))^-1}{}} with \eqn{s_p}{} as penalized first order derivative and \eqn{J_p}{} as penalized second order derivative. We begin with \eqn{v=0}{}. Not yielding a new maximum for a current v, we increase v step by step respectively bisect the step size. We terminate the iteration, if the step size is smaller than some reference value epsilon (eps=1e-3) without yielding a new maximum. We iterate for new parameter beta until the new log likelihood depending on the new estimated parameter beta differ less than 0.1 log-likelihood points from the log likelihood estimated before.\\
\end{Details}
\begin{Value}
\begin{ldescription}
\item[\code{Likelie}] corresponding log likelihood
\item[\code{step}] used step size
\end{ldescription}
\end{Value}
\begin{Author}\relax
Christian Schellhase <cschellhase@wiwi.uni-bielefeld.de>
\end{Author}
\begin{References}\relax
Penalized Density Estimation, Kauermann G. and Schellhase C. (2009), to appear.
\end{References}

