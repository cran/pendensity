\HeaderA{Abl2}{Calculating the second order derivative with and without penalty}{Abl2}
\keyword{math}{Abl2}
\begin{Description}\relax
Calculating the second order derivative of the likelihood function of the pendensity approach w.r.t. the parameter beta. Thereby, for later use, the programm returns the second order derivative with and without the penalty.
\end{Description}
\begin{Usage}
\begin{verbatim}
Abl2(penden.env, lambda0)
\end{verbatim}
\end{Usage}
\begin{Arguments}
\begin{ldescription}
\item[\code{penden.env}] Containing all information, environment of pendensity()
\item[\code{lambda0}] smoothing parameter lambda
\end{ldescription}
\end{Arguments}
\begin{Details}\relax
We approximate the second order derivative in ths approch with the negative fisher information. 
\deqn{J(\beta)= -  \frac{\partial^2 l(\beta)}{\partial \beta \ \partial \beta^T} \approx \sum_{i=1}^n s_i(\beta) s_i^T(\beta) .}{\eqn{J(beta)= partial^2 l(beta) / (partial(beta) partial(beta)) = sum(s[i](beta) s[i]^T(beta))}{}}
Therefore we construct the second order derivative of the i-th observation w.r.t. beta with the outer product of the matrix Abl1.cal and the i-th row of the matrix Abl1.cal.\\
The penalty is computed as \deqn{\lambda D_m}{\eqn{ lambda Dm}{}}.
\end{Details}
\begin{Value}
\begin{ldescription}
\item[\code{Abl2.pen}] second order derivative w.r.t. beta with penalty
\item[\code{Abl2.cal}] second order derivative w.r.t. beta without penalty. Needed for calculating of e.g. AIC.
\end{ldescription}
\end{Value}
\begin{Author}\relax
Christian Schellhase <cschellhase@wiwi.uni-bielefeld.de>
\end{Author}
\begin{References}\relax
Penalized Density Estimation, Kauermann G. and Schellhase C. (2009), to appear.
\end{References}

