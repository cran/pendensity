\HeaderA{marg.likelihood}{Calculating the marginal likelihood}{marg.likelihood}
\keyword{math}{marg.likelihood}
\begin{Description}\relax
Calculating the marginal likelihood.
\end{Description}
\begin{Usage}
\begin{verbatim}
marg.likelihood(penden.env,pen.likelihood)
\end{verbatim}
\end{Usage}
\begin{Arguments}
\begin{ldescription}
\item[\code{penden.env}] Containing all information, environment of pendensity()
\item[\code{pen.likelihood}] penalized log likelihood
\end{ldescription}
\end{Arguments}
\begin{Details}\relax
Calculating is done using a Laplace approximation for the integral of the marginal likelihood.

The needed values are saved in the environment.
\end{Details}
\begin{Value}
Returning the marginal likelihood.
\end{Value}
\begin{Author}\relax
Christian Schellhase <cschellhase@wiwi.uni-bielefeld.de>
\end{Author}
\begin{References}\relax
Penalized Density Estimation, Kauermann G. and Schellhase C. (2009), to appear.
\end{References}

