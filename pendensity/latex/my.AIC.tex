\HeaderA{my.AIC}{Calculating the AIC value}{my.AIC}
\keyword{math}{my.AIC}
\begin{Description}\relax
Calculating the AIC vaule of the density estimation. Therefore, we add the unpenalized log likelihood of the estimation and the degree of freedom, which are
\end{Description}
\begin{Usage}
\begin{verbatim}
my.AIC(penden.env, lambda0, opt.Likelihood = NULL)
\end{verbatim}
\end{Usage}
\begin{Arguments}
\begin{ldescription}
\item[\code{penden.env}] Containing all information, environment of pendensity()
\item[\code{lambda0}] penalty parameter lambda
\item[\code{opt.Likelihood}] optimal unpenalized likelihood of the density estimation
\end{ldescription}
\end{Arguments}
\begin{Details}\relax
AIC is calculated as
\eqn{AIC(\lambda)= - l(\hat{\beta}) + df(\lambda)}{}
\end{Details}
\begin{Value}
\begin{ldescription}
\item[\code{myAIC}] sum of the negative unpenalized log likelihood and mytrace
\item[\code{mytrace}] calculated mytrace as the sum of the diagonal matrix df, which results as the product of the invers of the penalized second order derivative of the log likelihood with the unpenalized second order derivative of the log likelihood
\end{ldescription}
\end{Value}
\begin{Author}\relax
Christian Schellhase <cschellhase@wiwi.uni-bielefeld.de>
\end{Author}
\begin{References}\relax
Penalized Density Estimation, Kauermann G. and Schellhase C. (2009), to appear.
\end{References}

