\HeaderA{new.lambda}{Calculating new penalty paramater lambda}{new.lambda}
\keyword{nonparametric}{new.lambda}
\begin{Description}\relax
Calculating new penalty parameter lambda.
\end{Description}
\begin{Usage}
\begin{verbatim}
new.lambda(penden.env,lambda0)
\end{verbatim}
\end{Usage}
\begin{Arguments}
\begin{ldescription}
\item[\code{penden.env}] Containing all information, environment of pendensity()
\item[\code{lambda0}] actual penalty parameter lambda
\end{ldescription}
\end{Arguments}
\begin{Details}\relax
Iterating for the lambda is stopped, when the changes between the old and the new lambda is smaller than 0.01*lambda0. If this criterion isn't reached, the iteration is terminated after 11 iterations.

The iteration formulae is
\deqn{\lambda^{-1}=\frac{\hat{\beta}^T D_m \hat{\beta}}{df(\hat{\lambda})-(m-1)}.}{\eqn{\lambda=(\beta^T D_m \beta)/(df(\lambda)-(m-1)}{}}
\end{Details}
\begin{Value}
Returning the new lambda.
\end{Value}
\begin{Author}\relax
Christian Schellhase <cschellhase@wiwi.uni-bielefeld.de>
\end{Author}
\begin{References}\relax
Penalized Density Estimation, Kauermann G. and Schellhase C. (2009), to appear.
\end{References}

