\HeaderA{f.hat}{Calculating the actual fitted values of y}{f.hat}
\keyword{nonparametric}{f.hat}
\begin{Description}\relax
Calculating the actual fitted values of the response, depending on the actual parameter set beta
\end{Description}
\begin{Usage}
\begin{verbatim}
f.hat(penden.env,ck.temp=NULL)
\end{verbatim}
\end{Usage}
\begin{Arguments}
\begin{ldescription}
\item[\code{penden.env}] Containing all information, environment of pendensity()
\item[\code{ck.temp}] actual weights, depending on the actual parameter set beta. If NULL, the beta parameter is caught in th environment
\end{ldescription}
\end{Arguments}
\begin{Details}\relax
Calculating the actual fitted values of the response, depending on the actual parameter set beta. Multiplying the actual set of parameters \eqn{c_k}{} with the base 'base.den' delivers the fitted values, depending on the group of covariates, listed in 'x.factor'.
\end{Details}
\begin{Value}
The returned value is a vector of the fitted value for each observation of y.
\end{Value}
\begin{Author}\relax
Christian Schellhase <cschellhase@wiwi.uni-bielefeld.de>
\end{Author}
\begin{References}\relax
Penalized Density Estimation, Kauermann G. and Schellhase C. (2009), to appear.
\end{References}

