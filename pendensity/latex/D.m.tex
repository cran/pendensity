\HeaderA{D.m}{Calculating the penalty matrix}{D.m}
\keyword{nonparametric}{D.m}
\begin{Description}\relax
calculating the penalty matrix depending on the number of covariates 'p', the order of differences to be penalized 'm', the corresponding difference matrix 'L' of order 'm', the covariate matrix 'Z', the number of observations 'n' and the number of knots 'K'.
\end{Description}
\begin{Usage}
\begin{verbatim}
D.m(penden.env)
\end{verbatim}
\end{Usage}
\begin{Arguments}
\begin{ldescription}
\item[\code{penden.env}] Containing all information, environment of pendensity()
\end{ldescription}
\end{Arguments}
\begin{Details}\relax
The penalty matrix is calculated as

\eqn{D_m=(L^T \otimes I_p) (I_{K-m} \otimes \frac{Z^T Z}{n}) (L \otimes I_p)}{}

The needed values are saved in the environment.
\end{Details}
\begin{Value}
Returning the penalty matrix.
\end{Value}
\begin{Author}\relax
Christian Schellhase <cschellhase@wiwi.uni-bielefeld.de>
\end{Author}
\begin{References}\relax
Penalized Density Estimation, Kauermann G. and Schellhase C. (2009), to appear.
\end{References}

