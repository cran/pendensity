\HeaderA{pen.log.like}{Calculating the log likelihood}{pen.log.like}
\keyword{nonparametric}{pen.log.like}
\begin{Description}\relax
Calculating the considered log likelihood. If one chooses lambda0=0, one gets the (actual) unpenalized log likelihood. Otherwise, one gets the penalized log likelihood for the used fitted values of the response y and the actual parameter set beta.
\end{Description}
\begin{Usage}
\begin{verbatim}
pen.log.like(penden.env,lambda0,f.hat.val=NULL,beta.val=NULL)
\end{verbatim}
\end{Usage}
\begin{Arguments}
\begin{ldescription}
\item[\code{penden.env}] Containing all information, environment of pendensity()
\item[\code{lambda0}] penalty parameter lambda
\item[\code{f.hat.val}] matrix contains the fitted values of the response, if NULL the matrix is caught in the environment
\item[\code{beta.val}] actual parameter set beta, if NULL the vector is caught in the environment
\end{ldescription}
\end{Arguments}
\begin{Details}\relax
The calculation depends on the fitted values of the response as well as on the penalty parameter lambda and the penalty matrix Dm.\\
\deqn{l(\beta)=\sum_{i=1}^{n} \left[ \log \{\sum_{k=-K}^K c_k(x_i,\beta) \boldsymbol\phi_k(y_i)\}  \right]- \frac 12 \lambda \beta^T D_m \beta}{\eqn{l(\beta)=sum(log(sum(c_k(x_i,\beta) \phi_k(y_i))))-0.5*\lambda \beta^T D_m \beta}{}}.

The needed values are saved in the environment.
\end{Details}
\begin{Value}
Returns the log likelihood depending on the penalty parameter lambda.
\end{Value}
\begin{Author}\relax
Christian Schellhase <cschellhase@wiwi.uni-bielefeld.de>
\end{Author}
\begin{References}\relax
Penalized Density Estimation, Kauermann G. and Schellhase C. (2009), to appear.
\end{References}

